\chapter{Deep Learning Frameworks}
Along with the deep learning methods in several domains, the deep learning frameworks have been developed to support the implementation. This chapter  presents several deep learning frameworks such as Caffe, Theano, TensorFlow, Torch, PyTorch,.... Besides, a comparison of 3 deep learning frameworks: Caffe, Theano, TensorFlow is studied. The study is detailed on two networks: AlexNet\cite{•} and a proposed network. The information that we want to study are not the speed, hardware utilization or extensibility; instead of, we want to know how the parameters effect to the result during training and choose the best framework for the next study. The comparative study of the frameworks is done on the CIFAR-10 dataset \cite{•} and the dataset of patches around the landmarks on the pronotum. 
\section{Deep learning frameworks}
Deep learning is more and more popular in many application domains over the last few year, there have a lot of laboratory (...) and industry group (Google, Facebook, Amazon) to develop the software frameworks. That thing is help easily to create and test the neural networks. At this time, many frameworks have been proposed for deep learning (see the complete list of deep learning software at DeepLearning.net \footnote{$http://deeplearning.net/software\_links/$} or DeepLearing4j.org \footnote{$https://deeplearing4j.org$}). In this section, we represent the main character of some of the widely used software frameworks: Caffe, Theano, TensorFlow, Torch; and the frameworks that used for other domain than computer vision, such as DyNet, DSSTNE.
\subsection{Caffe}
Caffe\cite{•} is a deep learning framework for computer vision. It is developed by Berkey AI Research (BAIR)\footnote{site} and community contributors. Caffe is written in C and C++. Besides providing a API on C++, it also provide the API on Python. Caffe is used by a large community of deep learning. Most of application which deployed in Caffe is processed on images. Caffe supports the train and test on both CPU and GPU.

\subsection{Theano}
Theano\cite{•} is a Python framework which developed by MILA lab at University of Montreal.It allows the user define, evaluate the convolutional neural network. With the purpose optimizes the 
compiler, Theano combines aspects of computer algebra system (CAS) which is particularly useful for the repeated tasks. Like Caffe, Theano is also providing the execution the network by CPU or GPU.
\subsection{TensorFlow}
TensorFlow\cite{•} is an open source software library for numerical computation using data flow graphs. Each node in the graph represents for a mathematical operation, while the edge represents the communication between the data arrays(tensors). Tensorflow supports Python and C++, along to allow computing distribution among CPU, GPU and even horizontal scaling.
\subsection{Torch}
Torch \cite{•} is a scientific computing framework for machine learning. It is written by Lua language. Torch 
\subsection{PyTorch}
\section{Experiments}
Table. shows the number of users of each frameworks in recently years.
Table. shows the properties of the deep learning frameworks. According the table, we have compared on some popular properties of a framework such as programming language core, training support devices (CPU, GPU), multi CPU (GPU) supporting, \ldots

To have a conclusion about the performance of the frameworks, we have set up on a single machine running on Ubuntu ....  For Caffe, version xxx is installed. For Theano and PyTorch, version xxx and yyy are used, respectively.

The experiments are done on two famous datasets: MNIST \cite{} and CIFAR-10 \cite{}. 
\subsection{MNIST dataset on Caffe and Theano}
\subsection{CIFAR-10 dataset on Caffe and Theano}

\section{Result and discussion}