\chapter{Using CNN to classify the landmark}
Deep learning strongly supports in image processing such as classification, recognition, object detection. In the context of this chapter, a Convolutional Neural Network is proposed to predict the landmark on the pronotum of beetle. The network has trained with a set of landmark patches. Then, we try to predict the location of a patch with the training result. 
\section{Data}
The pronotum dataset includes 293 images. For each image, a set of 8 landmarks have been set manually by the biologist. A patch around each landmark is extracted to put into the network.
The images have been divided into 3 sets: training set included 200 images ($200 \times 8 = 1600 $ patches), validation set had 60 images ($60 \times 8 = 480$ patches) and 33 images ($33 \times 8 = 264$ patches) belong to the test set.

To evaluate the network, the data which used by the network, has been chosen followed 3 ways: In first way, the images are ordered by the name and are putted into train, validation and test set, respectively; the second way, the test set is chosen firstly, then are train and validation, respectively; the last way, the images were divided randomly into 3 sets.

Besides, the patch's size also changed to find the suitable size for the patch for each landmark: $27 \times 27$, $45 \times 45$ and $63 \times 63$. The network trained with two types of the images: grayscale and RGB color.
\section{The network}
\section{Experiment}
\section{Result}