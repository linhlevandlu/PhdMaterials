\chapter{Machine Learning}
Machine learning is a norm refer to teach the computer the abilities which are only done by the humans. A machine learning algorithm is an algorithm that is able to learn from data. Most of machine learning algorithms can be divided into two categories: supervised learning and unsupervised learning algorithms. \\
A machine learning algorithm is built based on the tasked for a machine learning system. We have many kinds of task can be solved with machine learning. Some of common machine learning tasks include the following:
\begin{itemize}
	\item \textit{Classification}: In this type of task, the computer is asked to indicate a category in k category which the input belongs to. To solve this task, the learning algorithm uses a function $y=f(x)$, the model assigns the input described by vector $x$ to a category identified by score y.
	\item \textit{Classification without input}: A challenge of classification is missing the input vectors. In this case, to solve the classification task, the learning algorithm only has to define a single function mapping from a vector input to a category output. When some of inputs are missing, instead of providing a single classification function, the learning algorithm must learn a set of functions. Each function corresponds to classifying x with different subset of its inputs missing.
	\item \textit{Regression}: the computer program is asked to predict a numerical value given some input.
	\item \textit{Transcription}: machine learning system is asked to observe a relatively unstructured representation of some kind of data and transcribe it into discrete, textual form.
	\item \textit{Translation}: The input already contains the sequence of symbols in some languages, the computer program must convert it into the sequence of symbols of other languages.
	\item \textit{Structure output}: involve any task where the output is a vector with important relationships between the different elements.
	\item \textit{Anomaly detection}
	\item \textit{Synthesis and sampling}: The program is asked to generate the new example that are similar with the training data.
	\item \textit{Imputation of missing value}: The algorithm mus provide a prediction of the values of the missing entries in a new example.
	\item \textit{Denoising}
	\item \textit{Density estimation or probability mass function estimation}
\end{itemize}
\section{Supervised learning algorithms}
\section{Unsupervised learning algorithms}
\section{Stochastic Gradient Descent}