In the previous of thesis, we have studied the basic methods in image processing field. Then, the methods have been applied to determine the landmarks on biological images. They are known as the very well methods in image processing. In this part of thesis, we introduce another field that we have also applied to image analysis in recent years, that is machine learning, specially deep learning. 


Chapter 5 gives the principle context of machine learning. In this chapter, we present there problems of machine learning and we also give an overview about some machine learning algorithms.


Chapter 6 presents an sub-field of machine learning, \textbf{Deep Learning}. It displays an overview of deep learning problems. It has also present the convolutional neural network, a popular technique has been used in Deep Learning. At the end of the chapter, the libraries, which have been developed for Deep Learning, are also introducing.


Chapter 7 shows the applying of Deep Learing to predict the landmarks on biological images. This chapter presents the building process of the networks which have used to predict the landmarks. In this chapter also review some convolutional neural network, which have been employed to predict the facial keypoints.