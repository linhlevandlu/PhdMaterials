\chapter{Segmentation}
\section{Canny algorithm}
In 1986, \textbf{John F.Canny} had proposed a method to determine the edge in image. This is a technique to detect the useful structure of the object in digital image. Until now, the Canny algorithm\ref{canny1986computational} is used widely for the segmentation in computer vision. The process of Canny algorithm can be described in 4 steps as follows:
\begin{enumerate}
	\item Smoothing the image to reduce the noises by using Gaussian filter
	\item Finding the intensity and direction gradient of each pixel in image
	\item Eliminating the weak edge by using the edge thinning technique.
	\item Applying double threshold to determine the potential edges
\end{enumerate}
	\subsection{Gaussian filter}
	To smooth the image, a Gaussian filter is applied to convolve with the image. This step will help to reduce the effects of the noises on the edge detector. Normally, the equation of a Gaussian kernel with size $(2k+1)$ x $(2k + 1)$  is computed as:
	\begin{equation}
	H_{ij}=\frac{1}{2\pi\sigma^2}exp(-\frac{(i-(k+1))^2 + (j-(k+1))^2}{2\sigma^2});1\leq i,j \leq (2k+1)
	\end{equation}
	where $k$ is the size of kernel, and it should be a odd number.\\
	For example, a 3x3 Gaussian filter with $\sigma = 1 $ as followed:
	\begin{equation}
		G = 
		\begin{bmatrix}
		1 & 2 & 1\\
		2 & 4 & 2\\
		1 & 2 & 1		
		\end{bmatrix}
	\end{equation}	
	
	The selection of the size of the Gaussian kernel is important, it will affect the performance of the detector. If the size of the kernel is large, the detector can be sensitive to noise; otherwise, if the kernel's size is small, the detector can be destroy many strong edge. In the practice, this step is combined into Sobel convolution with a 3x3 kernel, which used to finding the intensity and direction gradients at each pixels of image.
	\subsection{Sobel convolution}
	The points belong to the edge in an image can stay in any direction, so the Canny algorithm uses four filters to detect the edges (vertical, horizontal and two diagonal edges) in the image. And the Sobel operator is used to detect the edges. This operator returns a value for the first derivative in horizontal direction $(G_x)$ and the vertical direction $(G_y)$. From these values, the gradient and direction of edge at each pixel are determined:
	\begin{equation}
		G = \sqrt{{G_x}^2 + {G_y}^2}
	\end{equation}
	\begin{equation}
		\phi = atan2(G_y,G_x)
	\end{equation}
	In this case, the kernel of Sobel convolution is 3x3, and it is also combined the Gaussian filter to smooth the image. The kernels are used to convolute the horizontal direction and vertical direction as follows:
	\begin{equation}
		G_x = 
		\begin{bmatrix}
		-1 & 0 & 1\\
		-2 & 0 & 2\\
		-1 & 0 & 1		
		\end{bmatrix}, 
		G_y = 
		\begin{bmatrix}
		-1 & -2 & -1\\
		0 & 0 & 0\\
		1 & 2 & 1		
		\end{bmatrix}
	\end{equation}
	
	The edge direction angle is rounded to one of four angles which were presented for four directions: vertical, horizontal, and two diagonals $0^o, 45^o, 90^o \text{ and } 135^o$.
	\subsection{Non-maximum suppression}
	Non-maximum suppression is applied to thin the edge in an image. Thus, this operation is used to suppress all the gradient values to 0 except the local maximal. At every pixel, it suppress the gradient value of the center pixels if its magnitude is smaller than the magnitude of one out of two neighbors in the gradient direction. In details:
	\begin{itemize}
	\item If the gradient direction angle is \textbf{0} degree, the point will be considered to be on the edge if the gradient magnitude is greater than the magnitude at pixels in the \textbf{east} and \textbf{west} directions.
	\item If the gradient direction angle is \textbf{45} degree, the point will be considered to be on the edge if the gradient magnitude is greater than the magnitude at pixels in the \textbf{north east} and \textbf{south west} directions.
	\item If the gradient direction angle is \textbf{90} degree, the point will be considered to be on the edge if the gradient magnitude is greater than the magnitude at pixels in the \textbf{north} and \textbf{south} directions.
	\item If the gradient direction angle is \textbf{135(-45)} degree, the point will be considered to be on the edge if the gradient magnitude is greater than the magnitude at pixels in the \textbf{north east} and \textbf{south west} directions.
	\end{itemize}
	\subsection{Double threshold}
	After applying the non-maximum suppression, the edges pixels are presented. However, there are still some edge pixels effected by noise. Double threshold will filter out the edge pixels with the weak gradient value and preserve the edge with the hight gradient value.
	\begin{itemize}
		\item A pixel called strong pixel (hence, it belong to the edge), if the edge pixel's gradient value is higher than the high threshold value.
		\item A pixel will be suppressed, if the edge pixel's gradient value is smaller than the low threshold value.
		\item A pixel called weak pixel (can be belong to the edge or not), if the edge pixel's gradient value is larger than low threshold value and smaller than high threshold value. A weak pixel can be belong to the edge if it connected with a strong pixel in 8-connected; else, it will be suppressed.
	\end{itemize}
	Thus, the accuracy of algorithm is depended on two parameters: the kernel of Gaussian filter and thresholds value. As said before, if we choose incorrect the kernel size of Gaussian filter, we can not reduce the noise or we can remove the real edge. Besides, the values of double threshold is also important to filter out the edge pixels. In practice, 1:3 is the good ratio between lower threshold  and upper threshold in Canny.
	\subsection{Summary}
	
\section{Suzuki algorithm}