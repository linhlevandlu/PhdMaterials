\documentclass[12pt,a4paper]{report}
\newcommand\tab[1][1cm]{\hspace*{#1}}
\usepackage{fullpage}
\usepackage[margin=2cm]{geometry}
\usepackage{mathtools,amsmath}
\usepackage{graphicx}
\usepackage[]{algorithm2e}
\usepackage{tikz}
\usepackage{multicol}
\usepackage{subfig}
\begin{document}
\title{Weekly report}
\author{Maire AIMAR-BEURTON and LE Van Linh}
\date{November 2016}
\maketitle
\begin{abstract}
	This document contains the summary about the morphometry and deep learning studied. Besides, it also contains the algorithms to apply for image processing such as segmentation or detection the dominant points.
\end{abstract}
\part{Morphometry}
Morphometry (or morphometrics) is a norm refers to the analysis of form, specifics size and shape of the object in digital image. Morphometry analyses are commonly performed on organisms, and are useful in analyzing the structure of the organisms. Morphometry can be used to extract the general information of creatures, or reconstruct the shape of the organism based on the analysed information that we had. Besides, morphometry can also detect the changes on creatures. Based on these information, we can statically the hypotheses about the factors that affect to the changes of shape.\\[0.2cm]
We have three distinct approaches are usually use: traditional morphometry, landmark-based morphometry and outline-based morphometry.
\begin{enumerate}
	\item Traditional morphometry:  measure the size of the object such as the length, width, angle, ratio and areas on object. The traditional morphomentry is using many measurement of size that most will be highly correlated; as a result, there are few independent variables despite the many measurements.
	\item Landmark-based morphometry: finding enough landmark to provide  a comprehensive description of shape. From the set of beginning landmarks, we can estimate the the missing landmarks.
	\item Outline-based morphometry: a mathematical functions is fitted to points sampled along the outline.
\end{enumerate}
In this part, we will describe the method to analysis the morphometry based on the landmarks. The method is through the several technique in image processing. At the end of this part, a software had built to verify the linkage of the steps.
	\chapter{Segmentation}
\section{Canny algorithm}
In 1986, \textbf{John F.Canny} had proposed a method to determine the edge in image. This is a technique to detect the useful structure of the object in digital image. Until now, the Canny algorithm\cite{canny1986computational} is used widely for the segmentation in computer vision. The process of Canny algorithm can be described in 4 steps as follows:
\begin{enumerate}
	\item Smoothing the image to reduce the noises by using Gaussian filter
	\item Finding the intensity and direction gradient of each pixel in image
	\item Eliminating the weak edge by using the edge thinning technique.
	\item Applying double threshold to determine the potential edges
\end{enumerate}
	\subsection{Gaussian filter}
	To smooth the image, a Gaussian filter is applied to convolve with the image. This step will help to reduce the effects of the noises on the edge detector. Normally, the equation of a Gaussian kernel with size $(2k+1)$ x $(2k + 1)$  is computed as:
	\begin{equation}
	H_{ij}=\frac{1}{2\pi\sigma^2}exp(-\frac{(i-(k+1))^2 + (j-(k+1))^2}{2\sigma^2});1\leq i,j \leq (2k+1)
	\end{equation}
	where $k$ is the size of kernel, and it should be a odd number.\\
	For example, a 3x3 Gaussian filter with $\sigma = 1 $ as followed:
	\begin{equation}
		G = 
		\begin{bmatrix}
		1 & 2 & 1\\
		2 & 4 & 2\\
		1 & 2 & 1		
		\end{bmatrix}
	\end{equation}	
	
	The selection of the size of the Gaussian kernel is important, it will affect the performance of the detector. If the size of the kernel is large, the detector can be sensitive to noise; otherwise, if the kernel's size is small, the detector can be destroy many strong edge. In the practice, this step is combined into Sobel convolution with a 3x3 kernel, which used to finding the intensity and direction gradients at each pixels of image.
	\subsection{Sobel convolution}
	The points belong to the edge in an image can stay in any direction, so the Canny algorithm uses four filters to detect the edges (vertical, horizontal and two diagonal edges) in the image. And the Sobel operator is used to detect the edges. This operator returns a value for the first derivative in horizontal direction $(G_x)$ and the vertical direction $(G_y)$. From these values, the gradient and direction of edge at each pixel are determined:
	\begin{equation}
		G = \sqrt{{G_x}^2 + {G_y}^2}
	\end{equation}
	\begin{equation}
		\phi = atan2(G_y,G_x)
	\end{equation}
	In this case, the kernel of Sobel convolution is 3x3, and it is also combined the Gaussian filter to smooth the image. The kernels are used to convolute the horizontal direction and vertical direction as follows:
	\begin{equation}
		G_x = 
		\begin{bmatrix}
		-1 & 0 & 1\\
		-2 & 0 & 2\\
		-1 & 0 & 1		
		\end{bmatrix}, 
		G_y = 
		\begin{bmatrix}
		-1 & -2 & -1\\
		0 & 0 & 0\\
		1 & 2 & 1		
		\end{bmatrix}
	\end{equation}
	
	The edge direction angle is rounded to one of four angles which were presented for four directions: vertical, horizontal, and two diagonals $0^o, 45^o, 90^o \text{ and } 135^o$.
	\subsection{Non-maximum suppression}
	Non-maximum suppression is applied to thin the edge in an image. Thus, this operation is used to suppress all the gradient values to 0 except the local maximal. At every pixel, it suppress the gradient value of the center pixels if its magnitude is smaller than the magnitude of one out of two neighbors in the gradient direction. In details:
	\begin{itemize}
	\item If the gradient direction angle is \textbf{0} degree, the point will be considered to be on the edge if the gradient magnitude is greater than the magnitude at pixels in the \textbf{east} and \textbf{west} directions.
	\item If the gradient direction angle is \textbf{45} degree, the point will be considered to be on the edge if the gradient magnitude is greater than the magnitude at pixels in the \textbf{north east} and \textbf{south west} directions.
	\item If the gradient direction angle is \textbf{90} degree, the point will be considered to be on the edge if the gradient magnitude is greater than the magnitude at pixels in the \textbf{north} and \textbf{south} directions.
	\item If the gradient direction angle is \textbf{135(-45)} degree, the point will be considered to be on the edge if the gradient magnitude is greater than the magnitude at pixels in the \textbf{north east} and \textbf{south west} directions.
	\end{itemize}
	\subsection{Double threshold}
	After applying the non-maximum suppression, the edges pixels are presented. However, there are still some edge pixels effected by noise. Double threshold will filter out the edge pixels with the weak gradient value and preserve the edge with the hight gradient value.
	\begin{itemize}
		\item A pixel called strong pixel (hence, it belong to the edge), if the edge pixel's gradient value is higher than the high threshold value.
		\item A pixel will be suppressed, if the edge pixel's gradient value is smaller than the low threshold value.
		\item A pixel called weak pixel (can be belong to the edge or not), if the edge pixel's gradient value is larger than low threshold value and smaller than high threshold value. A weak pixel can be belong to the edge if it connected with a strong pixel in 8-connected; else, it will be suppressed.
	\end{itemize}
	Thus, the accuracy of algorithm is depended on two parameters: the kernel of Gaussian filter and thresholds value. As said before, if we choose incorrect the kernel size of Gaussian filter, we can not reduce the noise or we can remove the real edge. Besides, the values of double threshold is also important to filter out the edge pixels. In practice, 1:3 is the good ratio between lower threshold  and upper threshold in Canny.
	\subsection{Summary}
	With applying double thresholding in the last stage, Canny had provied a strict condition to consider the weak edge as well as remove the pixels which were not belong to the edge. So far, Canny algorithm is good method to determined the edge in image, it is used by many application in image processing.
\section{Suzuki algorithm}
The Canny algorithm had detected the edge in the image. We can apply many difference methods to track the edge. \textbf{S. Suzuki} and \textbf{K. Abe}\cite{suzuki1985topological} had proprosed a method to get the border of object in image. This method is based on the topological structure analysis on binary image.\\[0.2cm]
Following this method, it detects two kinds of border in image. The first is outer border, which is defined by a set of border points between an arbitrary 1-component and the 0-component which surrounds it directly; another type is hole border which refers to the set of border points between a hole and the 1-componet which surrounds the hole directly. In this case, the 1-component (or 0-component) is connected component of 1-pixels (or 0-pixels).\\[0.2cm]
In our case, our purpose is getting the edges which were detectecd by Canny\cite{canny1986computational}. An edge is consider as an outer border or a hole border does not important. So, the Suzuki algorithm could make some changes to fit with our aim. The processes of algorithm is desribed as follows:\\[0.2cm]
Let an input binary image is \textbf{$F=\{f_{ij}\}$}. Set initially $NBD = 1$ (denoted the sequence number of border.)
\begin{enumerate}
	\item Select one of the following:
		\begin{enumerate}
			\item If \textbf{$f_{ij} = 1$} and \textbf{$f_{i,j-1} = 0$}, increment NBD, $(i_2,j_2) \gets (i,j-1)$ (pixel $(i,j)$ is the starting point of an outer border).
			\item If \textbf{$f_{ij} \geq 1$} and \textbf{$f_{i,j+1} = 0$}, increment NBD, $(i_2,j_2) \gets (i,j+1)$ (pixel $(i,j)$ is the starting point of an hole border).
			\item Otherwise, go to step (3)
		\end{enumerate}
	\item From the starting point \textbf{(i,j)}, the process to trace the edge is done by substeps following:
		\begin{itemize}
			\item[2.1] Starting from point $(i_2,j_2)$, look around clockwise the pixels in the neighborhood (8-connected) of $(i,j)$ and find the first non-zero pixel $(i_1,j_1)$. If no non-zero pixel is found, assign -NBD to $f_{ij}$ and go to step (3)
			\item[2.2] $(i_2,j_2) \gets (i_1,j_1)$ and $(i_3,j_3) \gets (i,j)$
			\item[2.3] Starting from the \textbf{next element of the pixel $(i_2,j_2)$} in the counterclock-wise order, check the pixels neighborhood of current pixel $(i_3,j_3)$ to find the first non-zero pixel $(i_4,j_4)$.
			\item[2.4] Chang the value $f_{i_3,j_3}$ of the pixel $(i_3,j_3)$ as follows:
				\begin{enumerate}
					\item If the pixels $(i_3,j_3+1)$ is a 0-pixel examined in the substep (2.3) then $f_{i_3,j_3} \gets -NBD$. Else, $f_{i_3,j_3} \gets NBD$ unless $(i_3,j_3)$ is on an already border.
					\item If the pixels $(i_3,j_3+1)$ is not a 0-pixel examined in the substep (2.3) and $f_{i_3,j_3} = 1$ then $f_{i_3,j_3} \gets NBD$
					\item Otherwise, do not change $f_{i_3,j_3}$.
				\end{enumerate}
			\item[2.5] If $(i_4,j_4) = (i,j)$ and $(i_3,j_3) = (i_1,j_1)$ (coming back to the starting point), then go to step (3); otherwise, $(i_2,j_2) \gets (i_3,j3), (i_3,j_3) \gets (i_4,j_4)$ and go back to step (2.3)
		\end{itemize}
	\item Resume the scan from the pixel $(i,j+1)$. The algorithm is stop when the scan reaches the lower right corner of the image.
\end{enumerate}

	\chapter{Dominant points}
In shape analysis, extracting features from the curves is an important step because in another way, we can re-construct the shape from the features. The term dominant points, also called as siginficant points, points of interest, corner points or landmarks is assigned to the points which have the high effect on boundary of object; their dectection is a very important aspect in contours methods because these concentrate the information of a curve on the shape.\\[0.2cm]
Dominant points can be used to produce a presentation of a shape contour for futher processing. The representation ...
In the content of this chapter, we will discuss about the methods to determine the dominant in digital image.\\[0.2cm]
There are many approaches developed for detecting dominant points and the methods can be classified into three groups follows:
\begin{itemize}
	\item Dectermine the dominant points using some significat measure other than curvature
	\item Evaluate the curvature by transforming the contour to the Gaussian scale space.
	\item Search for dominant points by estimating directly the curvature in the original image space.
\end{itemize}
\section{Hough Transform}
One of the challenges in image processing is detecting the characteristic of the object for recognition. Shape recognition is done by searching or detecting a class of simple geometric object such as line, curves in the image and comparing with the model. The matching score of the shapes is calculating by a measurement distance (such as Bhattacharyya). Instead of comparing between the geometric classes from the shapes, we can detect the presence of a shape in anther shape by searching each feature of the shapes. To sovle this problem, Hough Transform (HT)\cite{mukhopadhyay2015survey} is varied. At the beginning, HT\cite{vc1962method} is used to detect the line. It converts the space of parameters from x and y (coordinate of points in line) to space of slope and y-intercept of the line by voting process. For each object statisfying with equation of a line, it votes for the bin have correspondence slope and y-intercept. The set of bins is called the accumulator.
\subsection{Generalizing Hough Transform}
Until now, HT is still a good method for line detection or object recognition. But one of the weakness of HT is cannot determine the end points of the line segments. For this reason, the Generalized Hough Transform (GHT), introduced by Ballard\cite{ballard1981generalizing} is a generalization of HT to detect non-parametric curves. The process includes two phases: learning and recognition. In learning phase, a R-table is construct for model object. R-table is constructed based on the geometric information of each points in curves of object model with a reference point. The reference point can be arbitrary point in the model. Each row in R-table includes the gradient direction of each point which was chosen as index of table; and the polar coordinate values of each point. This mean that a gradient direction can be having many polar coordinate values. During recognition phase, an accumulator is created, called Hough Space. For each point in the scene object, finding the correspondce gradient direction in the R-table and voting at all the coordinate values. The peak in accumnulator is position of reference point of the model object in the scene object. And the peak value is equal to the number of boundary points of the object  when the model and the scene match perfectly.\\[0.2cm]
During recognition phase, the translation and rotation between model object and scene object is determine by principal component axis. Based on the curve points of the object, the centroid of each object is calculated. Then, the principal axis of each object is indicated. The translation between two objects is difference of two centroid points. The angle to rotate is the angle difference of two axes.\\[0.2cm]
When model and scene are matching, the dominant points (landmarks) of scene object is estimated from landmarks of model object by applying the translation, rotation from the centroid points. The last result is verifying by apply template matching (which will discuss as section \ref{template}).
\subsubsection{Result}
Using GHT to extract the landmarks on beetle is experiment on 287 images of right mandible of beetle. To compare the matching between the location of manual landmarks and estimated landmarks, the centroid size is compute for each set of landmarks.\\[0.2cm]
\begin{figure}[h!]
\centering
\subfloat[Model 28]{\label{figpca1}\includegraphics[width=0.5\textwidth]{./images/cmodel28}}~~
\subfloat[Model 71]{\label{figpca2}\includegraphics[width=0.5\textwidth]{./images/cmodel71}}
\caption{The accuracy of centroid size on mandible}
\label{figpca}
\end{figure}
Figure \ref{figcentroidSize} display the accuracy of the centroid size of estimated landmarks when we compare with the centroid size of manual landmarks. We use 2 images (Md28.JPG and Md71.JPG) as model. The number of landmarks be detected on scene object is 100\%. For model 28, 71.43\% the centroid of estimated landmarks is placed inside the standard deviation (SD) of manual landmarks, 28.57\% is outside the SD. The ratio for model 71 are 70.03\% and 29.97\%.
\subsection{Probabilistic Hough Transform}
To speed up Hough Transform, instead of processing on all data set, we can consider a subset of data points. The popular method is  \textbf{Probabilistic Hough Transform}. Probabilistic Hough Transform (PHT) is used to detect the presence of a model image in a scene image based on the group of features. The hypothesised location of the model image in the scene image is indicated based on the conditional probability that any pair scene lines agreement about a position in model image. Applying PHT can be separated into two steps: firstly, recording the information of model image and try to find the presence of the model image in scene image (called training process); secondly, predicting the pose of model image in the scene image (called estimating process).\\[0.2cm]
During training process, choose an arbitrary point in the model image, called reference point. For each pair of lines in model image, the perpendicular distance and angle from each line to reference point is recording (angle is calculated as angle between line and a horizontal line begin from reference point). The presence of model image in scene image is detected by PHT with \textit{``vote"} procedure. Finally, we choose the similar pair lines between model image and scene image. The chosen pair is obtained from best \textit{vote} when we consider each pair of line in scene image with each pair of lines in model image.\\[0.2cm]
In estimating process, the reference point in model image is estimated in scene image by extending the perpendicular lines of the pair of scene lines at the appropriate position. There, we can estimate the pose of the model in the scene image.

\section{Template matching}\label{template}
Template matching is a technique for finding areas of an image that match to a template image (template) by sliding the template over each pixel on the image (commonly cross-correlation). At each position, the sum of products between two images is calculated. The position is considered similar if the sum value at this position is maximal. The equation of cross-correlation is as follows:
\begin{equation}
\label{eq:cross-correlation}
	R_{ccorr}(x,y) = \sum\limits_{x',y'}[T(x'.y').I(x + x', y + y')]
\end{equation}
Where:
\begin{itemize}
\item T is template which use to slide and find the exist in other image.
\item I is image which we expect to find the template image
\item $(x', y')$ are coordinates in template where we get the value to compute.
\item $(x + x', y + y')$ are coordinates in image where we get the value to compute when template $T$ sliding.
\end{itemize}
However, if we use the original image to compute and find the similarity, the brightness of the template and the image might change the conditions and the result. So, we can normalize the image before applying the cross-correlation to reduce the effect of lighting difference between them. The normalization coefficient is:
\begin{center}
\begin{equation}\label{eq:normalizeCoff}
Z(x,y) = \sqrt{\sum\limits_{x',y'}T(x'.y')^{2}.\sum\limits_{x',y'}I(x + x', y + y')^{2}}
\end{equation}
\end{center}
The value of this method when we normalized computation as below:
\begin{center}
\begin{equation}\label{eq:cross-correlation}
R_{ccorr\_norm}(x,y) =\frac{R_{ccorr}(x,y)}{Z(x,y)} = \frac{\sum\limits_{x',y'}[T(x'.y').I(x + x', y + y')]}{\sqrt{\sum\limits_{x',y'}T(x'.y')^{2}.\sum\limits_{x',y'}I(x + x', y + y')^{2}}}
\end{equation}
\end{center}
\section{Image registration}
Image registration is process of transforming difference data sets into the same space and comparing or integrating the data from them. The object in image registration may be the images, time series or viewpoints. It is having many application in medical, military or satellites. In recent years, image registration is applied for both 2D and 3D objects with many methods. These methods may be classified following the characteristics of the input such as \textit{intensity-based and feature-based}, \textbf{transformation}, \textit{spatial and frequency},... In the context of this section, we want to discuss around the methods of linear transformations which include rotation, translation and scaling. Besides, we use these method to generate the general model from several objects or detect the landmarks on the object.
\subsection{Principal component analysis (PCA)}
Principal component analysis is computed based on principal directions of the datasets (model and scene). The input of this method is the list of points on curves of model and scene object (called model points and object points). The origin of the axes is centroid of all points on the curves. One of the axes is the line over the origin and having the minimum distance to all points in the curves; another axis is perpendicular axis with the first axis. The translation between two objects is different distance of centroid point coordinates; the rotation is different angle of two coordinate systems. The steps in PCA are followed:
\begin{itemize}
	\item Compute the centroid of model and scene object,
	\item Calculate the principal axes of model and scene,
	\item Compute the translation and rotation
	\item Translate the model to the scene that they have the same centroid.
	\item Rotate the model followed the different angle to match with the scene.
\end{itemize}
\subsubsection{Result}
The method is experiment with the set of right mandibles. Most of model can be detected its position on the scene by PCA. It also determine the translation and rotation information (see figure \ref{figpca1}). But in the case the input has more the noises, the centroid may be missed with correct position, following it is wrong translation and rotation(see figure \ref{figpca2}). In these examples, the red line is presented for the scene object and blue points is presented for the model object, which we want to align with the scene.
\begin{figure}[h!]
\centering
\subfloat[PCA with less noises]{\label{figpca1}\includegraphics[width=0.4\textwidth]{./images/pca1}}~~
\subfloat[PCA with noises]{\label{figpca2}\includegraphics[width=0.4\textwidth]{./images/pca2}}
\caption{The result after applying the PCA}
\label{figpca}
\end{figure}
\subsection{Singular value decomposition (SVD)}
The PCA method is more effected by the noise, instead of using all the curve points, SVD just using a subset of points by optimal alignment between corresponding points of model and scene. Assume that \texttt{M} is a subset of the model points, \textbf{S} is a subset of the scene points and $p_i \in M, q_i \in S$ are two corresponding points. We would like to find the matrix transformed \textbf{R} so that the pair-wise distances between the corresesponding points is minimum. The pairwise distance is indicated by equation (\ref{eqpwdistance}).
\begin{center}
\begin{equation}\label{eqpwdistance}
	E = \sum_{i=1}^{n} {\|q_i - p_i^{'}\|}
\end{equation}
\end{center}
Where:
\begin{itemize}
	\item \textit{n}: is number of corresponding points
	\item \textit{$q_i$}: point of scene
	\item \textit{$p_i^{'}$}: point of model which corresponding with \texttt{$q_i$}
\end{itemize}
In detail, SVD method includes the following steps:
\begin{itemize}
	\item Calculate the cross covariance matrix: $M = P.Q^T$, where $P(Q)$ are matrix with i-th column is vector $p_i - c_T$ ($q_i - c_S$),
	\item Compute the singular value decomposition of matrix $M$: \textbf{$M = U.W.V^T$}. Where:
	\begin{itemize}
		\item U,V are \textit{m} x \textit{m} orthonormal matrices
		\item W is a diagonal \textit{m} x \textit{m} matrix with non-negative entries.
	\end{itemize}
	\item Indicate the orthonormal matrix (rotation matrix) $R = V.U^T$
\end{itemize}
\subsubsection{Result}
SVD method is solving the noise problem of PCA by using a set of corresponding points. The result obtained by applying the SVD also better PCA. But a disadvantage of SVD is requiring a accurate correspondences set of points which are usually not available.
\subsection{Iterative closest point (ICP)}
Based on the advantage and disadvantage of PCA and SVD. ICP combines two previous methods. The idea of ICP is using PCA to intial guess of correspondences and repeating SVD to improve correspondences. The steps of ICP are:
\begin{itemize}
	\item Transform the model by PCA aligment
	\item For each transformed model point, assign the closest scene point as its corresponding point. Align model and scene by SVD
	\item Repeat the step (2) until a termination criteria is met.
\end{itemize}
	\chapter{Software}
\section{The software architecture}
The architecture of program is followed 3-tier model. There-tier architecture is an architecture that each tier is designed, developed and maintained as independent. The advantage of this architecture is intended to allow any upgraded or replaced independent between the tiers. When user want to change the requirements or technology of a tier, it will non-affect to other tiers.\\[0.3cm]
The architecture of three-tiers includes:
\begin{itemize}
	\item \textbf{Data tier}: includes the classes which were designed for the data structure of program. It also provides the persistence mechanism to access the data.
	\item \textbf{Logic tier}: controls the functionality of application by performing detailed processing.
	\item \textbf{Presentation tier}: displays information related to user. It is a layer which received the require from user to program or return the result from program to user. 
\end{itemize}
\begin{figure}[h]
	\centering
	\includegraphics[scale=0.7]{images/software_3tiers}
	\caption{Three-tiers model}
	\label{fign3iters}
\end{figure}
\section{The modules}
The MAELab software mainly includes four modules: \textbf{segmentation}, \textbf{histograms}, \textbf{pht} and \textbf{correlation}. Besides, the software also includes the other modules to support for the main modules. The relation between the modules in the software is shown in figure \ref{fignsmodules}.
\begin{figure}[h]
	\centering
	\includegraphics[scale=0.5]{images/modules}
	\caption{Three-tiers model}
	\label{fignsmodules}
\end{figure}
The functions of each modules is describing as followed:
\begin{itemize}
	\item \textbf{io} module: Implement the functions to read and write file. It includes the \textbf{JPEG library} that used to decode and encode the JPEG image.
	\item \textbf{imageModel} module: Represent the data structure of the image.
	\item \textbf{segmentation} module: Implement the segmentation methods on image.
	\item \textbf{histograms} module: Contains the methods to compute the geometric histogram of the image.
	\item \textbf{pht} module: Describe the probabilistic hough transform duration.
	\item \textbf{correlation} module: Includes the template matching methods.
	\item \textbf{pointInterest} module: Combine the result of the modules such as segementation, histograms,... to provide the adapter to other module or other software.
\end{itemize}
\section{The modules}
\section{The classes architecture}
\part{Deep learning}
	\chapter{Machine Learning}
Machine learning is a norm refer to teach the computer the abilities which are only done by the humans. A machine learning algorithm is an algorithm that is able to learn from data. Most of machine learning algorithms can be divided into two categories: supervised learning and unsupervised learning algorithms. \\
A machine learning algorithm is built based on the tasked for a machine learning system. We have many kinds of task can be solved with machine learning. Some of common machine learning tasks include the following:
\begin{itemize}
	\item \textit{Classification}: In this type of task, the computer is asked to indicate a category in k category which the input belongs to. To solve this task, the learning algorithm uses a function $y=f(x)$, the model assigns the input described by vector $x$ to a category identified by score y.
	\item \textit{Classification without input}: A challenge of classification is missing the input vectors. In this case, to solve the classification task, the learning algorithm only has to define a single function mapping from a vector input to a category output. When some of inputs are missing, instead of providing a single classification function, the learning algorithm must learn a set of functions. Each function corresponds to classifying x with different subset of its inputs missing.
	\item \textit{Regression}: the computer program is asked to predict a numerical value given some input.
	\item \textit{Transcription}: machine learning system is asked to observe a relatively unstructured representation of some kind of data and transcribe it into discrete, textual form.
	\item \textit{Translation}: The input already contains the sequence of symbols in some languages, the computer program must convert it into the sequence of symbols of other languages.
	\item \textit{Structure output}: involve any task where the output is a vector with important relationships between the different elements.
	\item \textit{Anomaly detection}
	\item \textit{Synthesis and sampling}: The program is asked to generate the new example that are similar with the training data.
	\item \textit{Imputation of missing value}: The algorithm mus provide a prediction of the values of the missing entries in a new example.
	\item \textit{Denoising}
	\item \textit{Density estimation or probability mass function estimation}
\end{itemize}
\section{Supervised learning algorithms}
\section{Unsupervised learning algorithms}
\section{Stochastic Gradient Descent}
	\chapter{Classification}
Classification is a most of important task in machine learning. In classification, a function is constructed to determine the category of the input. Generally, the model of classification as following:

The process of classification includes two steps:
\begin{enumerate}
	\item \textbf{Training}: Use the \textbf{training set} to learn what every object of a class looks like. This duration is called training a classifier or learning a model. The training set is a set with the objects which have labeled with specific category.
	\item \textbf{Evaluation}: To evaluate the quality of the classifier. We use a new set \textbf{(test set)} of the objects and try to ask the classifier predict the category of the object in the test set.
\end{enumerate}
In the content of this chapter, we will discuss about the classification techniques, especially, linear classification which technique has used more in neural network and deep learning.
\section{Nearest Neighbour Classifier}
\section{K-Nearest Neighbour Classifier}
\section{Linear Classification}
The linear classification has two main components: 
\begin{itemize}
	\item \textbf{Score function}:  which used to map the raw data to score of a category.
	\item \textbf{Loss function}: that quantifies the agreement between predict score and the truth category of the data.
\end{itemize}
The simplest function of a linear mapping is:
\begin{equation}
	y = f(x_i,W,b) = Wx_i + b
\end{equation}
Where:
\begin{itemize}
	\item $x_i$ is the raw data, \textit{example: an image}.
	\item $W$: a matrix parameter, called \textbf{weight} matrix
	\item $b$: vector, called \textbf{bias} vector
	\item $y$: score when consider the data $x_i$ belongs to a category.
\end{itemize}
	\chapter{Deep Network}

In recent years, deep learning has succeeded with many applications in different fields. In which, the neural network is the most popular method in deep learning to solve the problem of the high dimensions dataset. This chapter focuses on the discussion about the deep network and its parameters. Firstly, we overview the neural network, its basic components. Secondly, we focus on the main method to update the parameters of neural networks. Thirdly, we mention the objects of this method, dataset. In this part, we will describe the necessary to enlarge the data for the neural network. At the end of this chapter, we present the algorithms that are hired to optimize the learning process.

\section{Deep learning}
Deep learning is known as a part of machine learning. It includes the methods based on learning data representation by allowing the computation on the models that are composed of multiple layers. Each layer extracts the presentation of the input data from the previous layer and computes a new presentation as the input for the next layer. In the hierachy of a deep learning model, the higher layers of representation enlarge aspects of the input that is important for discrimination and suppress irrelevant variations. Each level of representations is corresponding to the different level of abstraction. Deep learning methods work on a large dataset using the backpropagation algorithm to improve the result after each step. The methods of deep learning have effectively improved the results in classification problems \cite{}, object recognition \cite{}, speech recognition \cite{} and other domains \cite{}. 

In deep learning, using neural networks is known as the most popular method. This is a computing-system based on a collection of connected units (called \textit{neurons}). Each connection (called \textit{synapse}) between the neurons can transmit the signal from a neuron to another neuron. The receiving neuron processes the signal that it received, then it sends the resulting signal to another neuron connected to it. Neurons and synaptes may have the weights as learnabled variables, which can used to increase or decrease the strength of signal that it sends to next units. Normally, neurons are organized in layers with different kinds of transformation inside. The signal is travelled multiple times from the first layer (input layer) to the last layer (output layer).

\section{Neural network}
\subsection{Neural}
The basic components of the brain is a neuron. For the ordinary man, we have billion neurons in the human nervous system, and they are connected by the billion of synapses. Each neuron receives input signals from its dendrites and procedures output signals along its axon. At each neuron, the input signals are received, analysed and then given a decision. Fig. \ref{fignneuron} shows the model of a neuron: the left side presents the connections of a biological neuron in human brain, the right side describes the mathematical model at the neuron.

\begin{figure}[h]
	\centering
	\includegraphics[scale=0.5]{images/neurons.png}
	\caption{A drawing of a biological neuron and its mathematical model}
	\label{fignneuron}
\end{figure}

In the computational model of a neuron, the signals travel along the axons interact multiplicatively with the dendrites of the other neuron based on the synaptic strength at the synapse. The synaptic strength are learnable and control the strength at influence or inhibitory of one neuron on another. In basic mode, the input signals are summed and compared with a threshold value (called \textit{firing rate}). If the sum is greater than threshold value, the neuron can ``fire", sending a spike along its axon. It look likes what we have seen in the real world, the human makes a decision when he take into acount all the conditions. The computation at a neural can be considered as following:
\begin{equation}
	Y = \sum(\text{weight } \ast \text{input}) + \text{bias}
\end{equation}

Depending on the values of the parameters, the output of \textbf{$Y$} can be from $-\infty$ to $+\infty$. So, how do we decide what does a neuron should do in a large range? Should it "fire" or not? Therefore, an ``activation function" has been added to check the value of $Y$ and decide what the neural should be ``fire" (called \textit{activated}) or not.

The first thing in usual is using a threshold for activation function. If the value of $Y$ is greater than the threshold value, the neural will be declared as actived; otherwise, it is not activated. This kind of activation is called \textit{Step} function. It seems that Step function is really work when we would like to create a binary classifier. \textit{For example}, we would like to classify the samples of a dataset includes the samples of two classes (i.e. Class 1 and Class 2). Clearly that Step function works well in this case because it just provides a value to precise a sample will belong to ``Class 1" or ``Class 2". But the problem does not stop there, the question is raised as what will happen if we want to classify more than two classes (i.e three or four classes)? Of course, Step function can still be used in this case if we consider an activated class and other classes are non-activated. But it is really hard to train and converge follows this way. It will be better if we have ``non-binary" activation which can provide the activated (or non-activated) probabilty for each class. The first solution is \textbf{Linear} function. This function gives a range of activations. The user can combine the output of some neurons before deciding. So, that is good too.

However, a neural can not stay alone, it need to connect to other neurons until the last one. In fact, the neurons are organized by the layers and they are connected. If each layer is activated by a linear funtion, the final activation function of the last layer is  just a linear function of the input of the first layer. This means the layers that we have tried to create can be replaced by a single layer. Therefore, we have to use some ``non-linear" activation function in the network if we would like to create a neural network with many layers, i.e. sigmoid, tanh or rectified linear Unit (ReLU).

\textbf{Sigmoid} function is one of the most widely used activation function (Eq. \ref{fsigmoid}). Its output is always going to be in range $(0,1)$.  It means this activation can bound the range of the output instead of $(-\infty, +\infty )$ of linear function. Consider on Eq. \ref{fsigmoid}, if $x$ stays in the range around the origin (i.e $[-2,2]$), then the value of activation has a change being considered. It means any small changes in the values of $x$ in this region will make the output to change significatly. Additional, when $x$ is bigger then $\sigma(x) \rightarrow 1$ and otherwise, $\sigma(x) \rightarrow 0$ if x smaller which means Sigmoid function makes clear distinctions on classification.

\begin{equation}
	\sigma(x) = \frac{1}{1+e^{-x}}
	\label{fsigmoid}
\end{equation}

\textbf{Tanh} is also another popular activation function (Eq. \ref{ftanh}). It is known as a scaled of Sigmoid function (two times Sigmoid). The characteristics of Tanh is similar with the Sigmoid function. However, the output of this functions is zero centered. It means the boundary range is changed to $(-1, 1)$. This point mj 

\begin{equation}
	tanh(x) = 2\sigma(2x) - 1 = \frac{2}{1 + e^{-2x}} - 1
	\label{ftanh}
\end{equation}

\textbf{Rectified Linear Unit (ReLU)} function (Eq. \ref{frelu}) is another activation function which has become in the last of few years. The activation is threshold at zero. At first look this function likes the linear function but in fact they are different because ReLU outputs zero across half its domain. This makes the derivatives through a ReLU remain large and consitent whenever the unit is active. Comparing with Sigmoid or Tanh functions, ReLU was found to greatly accelerate the update parameters of the networks. Addition, ReLU can be easily implemented by threshoding at zero.

\begin{equation}
	f(x) = max(0,x)
	\label{frelu}
\end{equation}

Besides, we have also some generalizations of ReLU such as \textbf{Leaky ReLU} (Eq. \ref{fleakyrelu}), \textbf{PReLU} (Eq. \ref{fPrelu}). They are also more broadly applicable.

\begin{equation}
	f(x) = max(0.01x, x)
	\label{fleakyrelu}
\end{equation}

\begin{equation}
	f(\alpha,x) = max(\alpha x, x)
	\label{fPrelu}
\end{equation}

Actually, we have many kinds of activation functions along with the mentioned functions above. Choosing an activation function for a neuron is depending on the objective of the user when they designed the network. For example, a sigmoid function works well for a classifier because approximating a classifier function as combinations of sigmoid is easier than ReLU for example.

\subsection{Neural network}

\begin{figure}[h]
	\centering
	\includegraphics[scale=0.5]{images/neural_net}
	\caption{A model of neural networks}
	\label{fignnnetworks}
\end{figure}

Fig. \ref{fignnnetworks} shows architecture of a simple neural network. The leftmost layer in this network is called the \textit{input layer}, the rightmost layer is called the \textit{output layer}. The neurons within the input layer are called input neurons, the neurons from output layer are called output neurons. When design the network, the input and the output are often straightforward. It means that the neural networks is designed where the output form one layer is used as the input to the next layer, there are no loops in the network, it always feed forward, never feed back (called feedforward networks).

So, the neural network includes many layers are designed as an directred acyclic graph from the intput to the output layer. The output of previous layer is used as the input of the next layer. At each layer excepts the output layer, the output is indicated by a activation function (i.e sigmoid, tanh,...). The size of a neural network can be to compute as the number of neurons, or the number of parameters.

\subsection{Deep network}

A deep neural network is a neural network with multiple layers between the input and the output layers. These layers are called hidden layers. Each layer tries to find the correct mathematical operator to turn its input into the next layer. The deep neural network forwards the data from the input layer to the output layer without looping back: The network creates the connections of neurons and assigns the ``weight" for each connection. At each layer, the weights and its input are multiplied and return an output to transfer to the next layer. Further, an algorithm is used to adjust the weights so that make certain parameters more influential until it receives the correct mathematical manipulation on all dataset. Fig. \ref{figndeepnetworks} presents a deep neural network with multiple hidden layers between the input and the output.

\begin{figure}[h]
	\centering
	\includegraphics[scale=0.5]{images/deep_neural_network}
	\caption{A deep neural network with multiple hidden layers}
	\label{figndeepnetworks}
\end{figure}

As other machine learning model, designing a deep neural network to solve a problem must be specified an optimization procedure, a cost function and a model family. In deep learning, the gradient-based learning is widely applied because it focus on the difference between the linear models and neural networks. In linear models, the solvers used to train the linear models with global convergence guarantees; instead of, neural network uses gradient-based optimization to drive the cost function to a lowest value after each iteration (because neural network is usually trained by using iterative). Along with gradient-based optimization, a cost fucntion and the represetation of the output of model must be chosen.

A cost function of a neural network is used to compute the difference between the real data and the model's outputs. It is usually updated by each iteration of training process (or validation process). In most of case, we use the cross-entropy between the training data an the model's predictions as the cost function. It means that we define a distribution $p(y | x; \theta)$ and we simply use the priciple of maximum likelihood. But sometimes, we mergely predict some statistic of $y$ conditioned on $x$. The total cost functions of a neural network often combines a primary cost function with a regularization term to make the learning algorithm intend to reduce its generation error.

In our mind, we think that we can separate the choice of the cost function and the output units but in fact, they are related together. For example, if we want to use cross-entropy as the cost function, we need to choose the way to represent the output so that the computing is easy and cheapest. Depending on the solved problem, the output units are chosen to fit with it. For example, we can use the Sigmoid function for a binary classification problem; the Linear function for a transformation with no nonlinearity; the Softmax function for a classifier over \textit{n} different classes.
\section{Back propagation}
A feedforward neural network accepts an input \textbf{$x$} as the initial information, then it was propagated up to the hidden units at each layer and finally procedure an output \textbf{$\hat{y}$}. This is called \textit{forward propagation}. During training, forward propagation can continue until the cost is stable. The \textit{back propagation} algorithm receives the information from the cost (lost function) then flows backward through the network to compute the gradient. This process is repreatedly to discover the gradient for updateing the weights in an attempt to minimize the loss function.

The back-propagation in the form of an algorithm is written as followed:
\begin{enumerate}
	\item \textbf{Input} $x$, \textbf{network} with $L$ layers. Set the corresponding activation $a^1$ for the input layer
	\item \textbf{Feedforward}: For each layer $l = 2, 3, \ldots, L $ compute $z^l = w^la^{l-1} + b^l$ and activation $a^l = \sigma(z^l)$
	\item Compute the error vector: $\delta^L = \nabla_{a}C \odot \sigma'(z^L)$
	\item \textbf{Back-propagation the error}: For each layer $l = L-1, L-2,\ldots, 2$ compute $\delta' = ((w^{l+1})^T \delta^{l+1}) \odot \sigma'(z^l)$
	\item \textbf{Ouput}: the gradient of the cost function given by $\frac{\partial C}{\partial w^l_{jk}} = a^{l-1}_k \delta^l_j$ and $\frac{\delta C}{\delta b^l_j} = \delta^l_j$
\end{enumerate}


The term back-propagation refers only to the method for computing the gradient through recursive application of \textbf{chain rule}. For an arbitrary function $f$, the gradient can be computed from its set of variables whose derivative are desired and set of inputs. It means the process at each function can do independent. In the next, we will describe how to compute the gradient of a function $f(x)$ by using Chain Rule of Calculus which is used to compute the derivaives of functions formed by composing other functions whose derivatives are known.
Let consider a simple multiplication function: $f(x,y) = xy$. The partial derivative of this function as followed:
\begin{equation}
	f(x,y) = xy \rightarrow \frac{\partial f}{\partial x} = y \text{ and } \frac{\partial f}{\partial y} = x
\end{equation}

The objective of derivative is indicating the rate of change of the function with respect to the variable surrounding a small region near a particular point. For example, if $x = 4, y = -3 \rightarrow f(x,y) = -12$, then the derivatives on $x, y$ are:
\begin{equation}
	\frac{\partial f}{\partial x} = -3 \text{ and } \frac{\partial f}{\partial y} = 4
\end{equation}
It means that if we increase the values of $x$, the value of function $f$ will be decreased. Otherwise, if we increase the value of $y$, the ouput of function $f$ is also increased.
For an addition function, the derivatives are:
\begin{equation}
	f(x,y) = x + y \rightarrow \frac{\partial f}{\partial x} = 1 \text{ and } \frac{\partial f}{\partial y} = 1
\end{equation}
And for a MAX operation, the gradient is $1$ on the larger input and $0$ on other inputs:
\begin{equation}
	f(x,y) = max(x,y) \rightarrow \frac{\partial f}{\partial x} = 1 (x \geq y) \text{ and } \frac{\partial f}{\partial y} = 0 (y \geq x)
\end{equation}

Consider example in Fig. \ref{figbackex1}, it presents the function $f(x,y,z) = (x + y)z$. Let see how to apply the derivative to compute the backward pass.

\begin{figure}[h]
	\centering
	\includegraphics[scale=0.6]{images/back_ex}
	\caption{The computation graph of function $f$}
	\label{figbackex1}
\end{figure}
The \textit{green} values are the input to compute the forward pass. It is very easy to obtain the result: $f(x,y,z) = (x +y)z = (-2 + 5)(-4) = -12$. In the backward pass, the values are computed by applying derivative: At multiply gate, we can easy to compute the gradient of add gate and input $z$ are $-4$ and $3$, respectively. At the add gate, it takes the gradient and multiples it to all of the local gradient for its input. So, the gradient on both $x$ and $y$ are $1 \ast -4 = -4$. Clearly that if we increase the value of $x$ and $y$, the value of add gate will be decreased and it will turn the ouput of multiply gate increase. So, back-propagation can see as gates communicating to each other to make a change on their outputs (decrease or increase) and to make the final value higher.

\section{Data augmentation}

The main problem of machine learning is to make an algorithm that not only perform on the training data but also on new inputs. Many plans have been used in machine learning to reduce the test error but not expense the training cost (re-trainning). A solution is given that the algorithm should be trained on a large dataset where it can covers all the issues of the attented problem.

More of the same, a deep neural network will be better if is is trained with more data. But in practice, we do not always have a large data instead the amount of data is very limitted. A solution for this problem is to create the fake data and add it to the dataset. However, for different tasks, we may be need to apply the different augmentation methods. \textit{For example,} in classification problem, a classifier needs to take a pair of input $x$ and summary it with a single category $y$. This means that the task of a classifier is to be invariant when the input transforms. So, we can generate new $(x,y)$ pairs just by transforming the $x$ inputs in the training set. But this approach is not reality applicable with a \textit{density estimation} task.

Data augmentation has been a particularly effective technique for object recognition problem \cite{}, speech recognition \cite{}. With the operations like translating the images a few pixels some directions, it can often generate the new images. Even the other operations like rotating or scaling the images have also proven effective.

Noise injection can also be as another form of data augmentation \cite{}. For many classification and regression tasks \cite{}, they have proven that the neural network can be improved the robustness when we train them with random noise applied to their input. They are also inserted into the hidden units which can see as an augmentation at multiple levels of consideration. \cite{} shown that this approach can be highly effective provided that the magnitude of the noise is carefully tuned.

Data augmentation is considered as a part of machine learning algorithms. Usually, operations are generally applicable while other operations are specific to one application domain. So, choosing augmentation methods should be thoroughly examined before applying.

\section{Optimization problem}
The optimization for neural networks is a active area, it involves to deep learning in many contexts and one of all is finding the parameters of a neural network that significantly reduce the cost function and improve the accuracy of the algorithm. In the previous section, we have seen how to compute the gradient with back-propagation. They are used to perform the parameters of the network to reduce the cost of the model. There are several appoaches for performing the update. In this section, we present some established and common techniques which have used to optimize the neural networks.

In previous section, we have tried to calculate the gradient of the loss function. Based on that we can compute the best direction along which we should change our weights to guarante the direction of the stepest descent. This process will be repeated to evaluate the gradients and to perform the parameters updating. This procedure is called \textit{Gradient Descent}. The simplest form of this procedure is to change the parameters along the negative direction. Assuming a vector of parameters \textbf{$x$} and the gradient \textbf{$dx$}, the update has the form: 
\begin{equation}
	x = x - learning\_rate \ast dx \text{ (\textbf{learning\_rate} is a fixed constant)}
\end{equation}
A neural network can be trained on a dataset of hundreds of millions of examples. It seems that wasteful to compute the full loss of the network to perform a parameter. So, we can just apply the Gradient descent over \textbf{batches} of training data to achieve the faster convergence. This process is called \textbf{Stochastic Gradient Descent (SGD)}. Using SGD to update training of a minibatch of \textit{m} examples can be described in  Algorithm \ref{sgdalgorithm}.
\begin{algorithm}
	\caption{SGD update at training iteration $k$}
	\label{sgdalgorithm}
	\begin{algorithmic}
		\REQUIRE Learning rate $\epsilon$
		\REQUIRE Initial parameter $\theta$
		\WHILE{stopping criterion not meet}
			\STATE Sample a minibatch of $m$ example from training set $\{ x^1,x^2,\ldots, x^m \}$ with corresponding targets $y^{(i)}$
			\STATE Compute gradient estimate: $\hat{g} \leftarrow + \frac{1}{m} \nabla_\theta \sum_i L(f(x^i;\theta), y^i) $
			\STATE Apply update $\theta \leftarrow \theta - \epsilon \hat{g}$
		\ENDWHILE
	\end{algorithmic}
\end{algorithm}

In SGD algorithm, the learning rate is an important parameter. In previous, the learning rate is fixed as a constant but in practice, it is necessary to decrease the learning rate over time. Denote the learning rate at iteration $k$ as $\epsilon_k$. It is common to decay the learning rate linearly until iteration $\tau$:
\begin{equation}
	\epsilon_k = (1 - \alpha) \epsilon_0 + \alpha \epsilon_\tau
\end{equation}
Where $\alpha = \frac{k}{\tau}$ and after iteration $\tau$, it is common to leave $\epsilon$ constant. Usually, the learning rate is chosen by trial and error. When using the linear schedule, the parameters to choose are $\epsilon_0$, $\epsilon_\tau$, and $\tau$: $\tau$ parameter prefers to the number of iteration after that the learning rate will be decreased, $\epsilon_\tau$ indicates the dropping of the learning rate, the last problem is how to choose the value for initial learning rate $\epsilon_0$. If the learning is too large, the cost function often increases significantly. Otherwise, if the learning rate is too low, the learning process will be slow and learning may become stuck with a high-cost value. Experience indicates that the learning rate in the first $100$ iterations should be higher than the next iterations. Therefore, setting up the first learning rate along with a decreasing schedule should be considered together to obtain the best result.

Besides SGD, the method of momentum \cite{} is designed to accelerate learning. It accumulates an exponentially decaying moving average of past gradients and continues to move in their direction. Acording that, a variable role of velocity $\upsilon$ is introduced, it presents the direction and speed that the parameters move through the parameter space. The velocity is set to an exponentially decaying average of the negative gradient. The SGD algorithm with momentum is describe in Algorithm \ref{sgdm_algorithm}.

\begin{algorithm}
	\caption{SGD with momentum}
	\label{sgdm_algorithm}
	\begin{algorithmic}
		\REQUIRE Learning rate $\epsilon$, momentum parameter $\alpha$
		\REQUIRE Initial parameter $\theta$, initial velocity $\upsilon$
		\WHILE{stopping criterion not meet}
			\STATE Sample a minibatch of $m$ example from training set $\{ x^1,x^2,\ldots, x^m \}$ with corresponding targets $y^{(i)}$
			\STATE Compute gradient estimate: $\hat{g} \leftarrow + \frac{1}{m} \nabla_\theta \sum_i L(f(x^i;\theta), y^i) $
			\STATE Compute velocity update: $\upsilon \leftarrow \alpha \upsilon - \epsilon \hat{g} $
			\STATE Apply update $\theta \leftarrow \theta + \upsilon$
		\ENDWHILE
	\end{algorithmic}
\end{algorithm}

\textbf{Nesterov Momentum} is another version of the momentum on SGD. The difference between Nesterov momentum and standard momentum is where the gradient is evaluated after applying the current velocity. It looks like we add a correlation factor to the standard momentum. The complete Nesterov momentum is presented in Algorithm \ref{sgdNm_algorithm}.

\begin{algorithm}
	\caption{SGD with momentum}
	\label{sgdNm_algorithm}
	\begin{algorithmic}
		\REQUIRE Learning rate $\epsilon$, momentum parameter $\alpha$
		\REQUIRE Initial parameter $\theta$, initial velocity $\upsilon$
		\WHILE{stopping criterion not meet}
			\STATE Sample a minibatch of $m$ example from training set $\{ x^1,x^2,\ldots, x^m \}$ with corresponding targets $y^{(i)}$
			\STATE Apply interim update: $\tilde{\theta} \leftarrow \theta + \alpha \upsilon$
			\STATE Compute gradient: $\hat{g} \leftarrow + \frac{1}{m} \nabla_{\tilde{\theta}} \sum_i L(f(x^i;\tilde{\theta}), y^i) $
			\STATE Compute velocity update: $\upsilon \leftarrow \alpha \upsilon - \epsilon \hat{g} $
			\STATE Apply update $\theta \leftarrow \theta + \upsilon$
		\ENDWHILE
	\end{algorithmic}
\end{algorithm}

The momentum algorithm can accelerate learning and mitigate the issues of the network, but it is expensive of including other hyperparameters. In which, the learning rate is one of the hyperparameters that is the most difficult to set because it has a significant impact on model performance. Tuning the learning rate for each trial is a very expensive process, so adjusting and adapting the learning rate throughout the training phase can be suitable to impale the cost because it is often highly sensitive to the directions in the parameter space. In this part, we highlight some common adaptive methods on learning rate.

\textbf{Delta-bar-delta} algorithm \cite{} is the first approach to adaptive local learning rates for model parameters during training. The algorithm describes that we should increase the learning rate when the partial derivative of the loss with respect the parameters remains the same sign and otherwise, if the partial derivative of the loss with respect the parameters change sign, then the learning rate should decrease. 

\textbf{AdaGrad} algorithm (Algorithm \ref{AdaGrad_algorithm}) adapts the learning rates by scaling them inversely proportional to the square root of the sum of all of their historical squared values. The parameters with the largest parital derivative of the loss have a correspondingly rapid decrease in their learning rate, while parameters with small partial derivatives have a relatively small decrease in their learning rate.

\begin{algorithm}
	\caption{The AdaGrad algorithm}
	\label{AdaGrad_algorithm}
	\begin{algorithmic}
		\REQUIRE Global learning rate $\epsilon$
		\REQUIRE Initial parameter $\theta$
		\REQUIRE Small constant $\delta$ ($i.e. 10^{-7}$)
		\STATE Initialize gradient accumulation variable r = 0
		\WHILE{stopping criterion not meet}
			\STATE Sample a minibatch of $m$ example from training set $\{ x^1,x^2,\ldots, x^m \}$ with corresponding targets $y^{(i)}$
			\STATE Compute gradient: $g \leftarrow + \frac{1}{m} \nabla_{\theta} \sum_i L(f(x^i;\theta), y^i) $
			\STATE Accumulate squared gradient: $r \leftarrow r + g \odot g$
			\STATE Compute update: $\Delta \theta \leftarrow - \frac{\epsilon}{\delta + \sqrt{r}} \odot g$ (Division and square root applied element-wise)
			\STATE Apply update: $\theta \leftarrow \theta + \Delta \theta$
		\ENDWHILE
	\end{algorithmic}
\end{algorithm}

\textbf{RMSProp} algorithm \cite{} (Algorithm \ref{RMSProp_algorithm}) modifies AdaGrad to perform better in non-convex function by chaning the gradient accumulation into an exponentially weighted moving average. 

\begin{algorithm}
	\caption{The RMSProp algorithm}
	\label{RMSProp_algorithm}
	\begin{algorithmic}
		\REQUIRE Global learning rate $\epsilon$, decay rate $\rho$
		\REQUIRE Initial parameter $\theta$
		\REQUIRE Small constant $\delta$ ($i.e. 10^{-6}$)
		\STATE Initialize gradient accumulation variable r = 0
		\WHILE{stopping criterion not meet}
			\STATE Sample a minibatch of $m$ example from training set $\{ x^1,x^2,\ldots, x^m \}$ with corresponding targets $y^{(i)}$
			\STATE Compute gradient: $g \leftarrow + \frac{1}{m} \nabla_{\theta} \sum_i L(f(x^i;\theta), y^i) $
			\STATE Accumulate squared gradient: $r \leftarrow \rho r + (1 - \rho)g \odot g$
			\STATE Compute parameter update: $\Delta \theta \leftarrow - \frac{\epsilon}{\delta + \sqrt{r}} \odot g$ (Division and square root applied element-wise)
			\STATE Apply update: $\theta \leftarrow \theta + \Delta \theta$
		\ENDWHILE
	\end{algorithmic}
\end{algorithm}

\textbf{Adam} algorithm \cite{} (Algorithm \ref{Adam_algorithm}) is another adaptive learning rate optimization. It looks like RMSProp with momentum but in Adam, the momenum is incorporated directly as an estimate of the first order moment of the gradient. Additional, it includes bias corrections to the estimates of both the first-order moments and the second order moments to account for their initialization at the origin.

\begin{algorithm}
	\caption{The Adam algorithm}
	\label{Adam_algorithm}
	\begin{algorithmic}
		\REQUIRE Step size $\epsilon$ (default suggestion $0.001$)
		\REQUIRE Exponential decay rates for moment estimates, $\rho_1, \rho_2$ in $[0, 1)$ (default suggestion $0.9 \text{ and } 0.999$ resprectively)
		\REQUIRE Small constant $\delta$ ($i.e. 10^{-8}$)
		\REQUIRE Initial parameter $\theta$
		\STATE Initialize $1^{st}$ and $2^{nd}$ moment variables $s = 0$, $r = 0$
		\STATE Initialize time step t = 0
		\WHILE{stopping criterion not meet}
			\STATE Sample a minibatch of $m$ example from training set $\{ x^1,x^2,\ldots, x^m \}$ with corresponding targets $y^{(i)}$
			\STATE Compute gradi ent: $g \leftarrow + \frac{1}{m} \nabla_{\theta} \sum_i L(f(x^i;\theta), y^i) $
			\STATE $t \leftarrow t + 1$
			\STATE Update biased first moment estimate: $s \leftarrow \rho_1 s + (1 - \rho_1)g$
			\STATE Update biased second moment estimate: $r \leftarrow \rho_2 r + (1 - \rho_2)g \odot g $
			\STATE Correct bias in first moment: $\hat{s} \leftarrow \frac{s}{1-\rho_1^t}$
			\STATE Correct bias in second moment: $\hat{r} \leftarrow \frac{r}{1 - \rho_2^t}$
			\STATE Compute parameter update: $\Delta \theta \leftarrow - \epsilon \frac{\hat{s}}{\sqrt{\hat{r} + \delta}} \odot g$ (Division and square root applied element-wise)
			\STATE Apply update: $\theta \leftarrow \theta + \Delta \theta$
		\ENDWHILE
	\end{algorithmic}
\end{algorithm}

In this section, we have described a highlight algorithms related to optimize the deep models by adapting the learning rate for each model parameter. The question is what algorithm should we choose? There is currently no consensus on this point. Choosing an algorithm is depending on the user's familiarity with the algorithm. In practice, the most popular algorithms are used include  SGD, SGD with momentum, RMSProp, and Adam.

\section{Conclusion}
In this chapter, we have described the basic components of deep learning by using the neural network. Using any algorithms for deep learning is depending on the users but generally, the problems that we have mentioned are indispensable \textit{i.e.} designing the algorithm, optimization problem. This chapter just provides to us a generally understood about deep learning but it is really important to help us have the first overviews about a method in this field before going to the further. In the next, we will enter into another method on deep learning, Convolutional Neural Network, and then is its application to predicting landmark on the beetle dataset.
	\chapter{Convolutional Neural Network}
Convolutional Neural Networks (CNNs) are similar with the original of Neural Networks. It means that CNNs has also the score function and the loss function at the end of network. Neural Networks receive an input and pass it through a series of hidden layer but in CNNs is deeper. Each hidden layer is made from a set of neurons, where each neuron is full connected with all neurons of previous layer. The layers of the CNNs have neurons arranged in 3 dimensions: \textbf{width, height, depth}. CNNs transform the original image layer by layer from the original pixel value to the final class score. In CNNs, some layers contain the parameters but other don't. This chapter will describe the architecture and the detail of each layer in the CNNs.
\section{Architecture}
A CNN is made from the layers. The common layers in CNN are convolutional, nonlinear, pooling and full connected layers. CNN takes image as an input, pass it through the series of layers and get an ouput. Each layer has a difference function to transform the input to another layer. 
\begin{figure}[h]
	\centering
	\includegraphics[scale=0.45]{images/cnn_architecture}
	\caption{An architecture of convolutional neural network}
	\label{figlncex}
\end{figure}~\\
\subsection{Convolutional layer}
Convolutional (CONV) layer computes a dot product between their weights and a small region in the input image for each small region in the input. At the output of neurons is combining the result  of the connected to local regions.\\[0.2cm]
CONV layer uses a set of learnable filters as parameters. Each filter is small spatially but extends the depth fo the input. During the forward pass, the filter is slided over each pixel of the input (from left to right, top to bottom) and calculate dot product between the entries of the filter and the input at this position. During the process, we can see the response of input for each filter such as the orientation of the edge or a blotch of some color on the firt layer. With an entire set of filters in each CONV layer, we will stack these activation maps along the depth dimension and procedure the output volume.\\[0.2cm]
Instead of connecting a neurons to all neurons in the previous layer, CONV connect each neurons to only a local region of the input. The spatial extent of this connectivity is a hyperparameter called the receptive field of the neuron (equal with the filter size). This extent has the depth axis is equal to the depth of the input. For example, if the input has size [32x32x3] and the filter size is [5x5] then each neuron in the CONV layer will have the weights to a [5x5x3] region in the input, and total of $5*5*3 = 75$ weights. This is the way that each neuron in CONV layer connected to the input; but how many neurons that we have in the output and how the order between the neurons. With 3 hyperparameters \textbf{depth, stride} and \textbf{zero-padding} will help us control the size of the CONV output.
\begin{itemize}
	\item \textbf{Depth}: corresponds to the number of filters we would like to use, each learning to look for something different in the input.
	\item \textbf{Stride}:  which we slide the filter. When the stride is 	1 then we move the filters one pixel at a time. If the stride is 2 (or more), then the filter will jump 2 (or more) at a time when we slide the filter.
	\item \textbf{Zero-padding}: pad the input with zeros arund the border.
\end{itemize}
\subsection{Pooling layer}
Pooling layer performs a downsampling operation along the spatial dimensions (width, height).
\subsection{Full connected layer}
Full connected layer computes the class scores of the input.
\section{Caffe framework}
\part{Conclusion}
	\include{includes/discussion}
	\include{includes/conclusion}
\bibliographystyle{unsrt}
\bibliography{includes/references}
\end{document}